\chapter{Grundlagen}\label{ch:grundlagen}

In diesem Kapitel sollen zunächst wichtige theoretische Grundlagen erklärt werden, die zum Verständnis der in den nachfolgenden Kapiteln beschriebenen Projektbearbeitungsschritte und -entscheidungen notwendig sind.

\section{Swift und SwiftUI}

Eine Festlegung, die bereits vor der Reimplementation der Apps getroffen wurde, ist die Wahl von Swift als Hauptprogrammiersprache der zu erstellenden iOS App. Die Programmiersprache wurde 2014 von Apple veröffentlicht und ersetzt seitdem Objective-C als die von Apple empfohlene Objektorientierte Sprache zur Erstellung von Applikationen für das Apple-Ökosystem mit den Betriebssystemderivaten Mac OS, iOS und bspw. auch Watch OS (Apple Watch). In der aktuellen Version 5 setzt Swift vielseitig etablierte Konzepte aus modernen Programmiersprachen um und ist hierbei nach dem subjektiven Empfinden des Projektteams leicht verständlich, schnell zu erlernen, sowie sehr kompakt. Zusammen mit Swift 5 wurde SwiftUI als deklaratives UI-Framework 2019 veröffentlicht, ergänzend zum herkömmlichen Constraint-basierten Layouting über UIView(Controller) und XIB/Storyboard-Dateien. SwiftUI orientiert sich an Frameworks wie React, bei denen Ansichten im Code definiert werden können, wodurch die Erstellung und Modifikation von Ansichten direkt im Editor geschehen kann, ohne die Notwendigkeit von etwaiger Controller-Logik, die UI-Elemente mit Code-Elementen (z.B. über die Annotation @IBAction) verbindet. Dies vereinfacht den Implementationsprozess, bringt jedoch auch verschiedene neue Architektur- und Datenflusskonzepte mit sich, die zunächst verstanden werden müssen, um SwiftUI effektiv und zielorientiert nutzen zu können.

\section{Datenfluss- und Architekturpatterns in SwiftUI}

\subsection{Darstellung von View-Daten}

Im Unterschied zum imparativen UI-Programmierkonzept, bei dem konkrete Elemente der Ansicht, wie bspw. Labels oder Textfelder imparativ konfiguriert und mit den darzustellenden Daten ausgestattet werden, referenziert eine SwiftUI Ansicht (\enquote{View}) die darzustellenden Daten direkt aus dem Code, da sie selbst deklarativ in Form von Code geschrieben werden kann.

\begin{figure}[H]
\includegraphics[width=\linewidth, bb=0 0 396 125]{swiftui.pdf}
\caption{Zwei wesentliche Arten des Datenflusses zwischen View und den in der View dargestellten Daten: unidirektional vs. bidirektional.}\label{fig:swiftui}
\end{figure}

\Cref{fig:swiftui} zeigt die zwei wesentlichsten Varianten, mithilfe derer Daten in der Ansicht dargestellt werden können: \textbf{unidirektional} und \textbf{bidirektional}. Bei der unidirektionalen Darstellung der Daten werden bei der Initialisierung der View die Daten im Konstruktor der View mitgegeben - da SwiftUI Views so genannte Structs sind, können die Daten im Verlauf der Zeit nicht mehr ändern (Struct-Felder sind \enquote{immutable}). SwiftUI rendert die assoziierten Elemente der View einmalig und verwendet diese nachfolgend zur Darstellung. Für viele Anwendungsfälle, in \Cref{fig:swiftui} anhang eines \enquote{Favorisieren-Buttons} illustriert, ist es jedoch notwendig, dass durch die View die dahinter liegenden Daten modifiziert werden. Hierfür wird der Property Wrapper \texttt{@State} bereitgestellt. Attribute, die hiermit ausgestattet werden, können während der Präsentation der View verändert werden. Eine Besonderheit hierbei ist, dass SwiftUI genau observiert, welche UI-Elemente durch das annotierte Attribut modifiziert oder konfiguriert werden, um bei einer Änderung des Attributs die entsprechenden Elemente neu zu rendern. Wird ein Datenattribut an eine Unteransicht weitergegeben, welche bspw. eine Detailansicht zu einem Objekt darstellt, dann wird das Attribut in der Unteransicht als \texttt{@Binding} referenziert.

\subsection{Coordinator-Pattern}

Um die Geschäftslogik der Anwendung von der Darstellungslogik weitestgehend zu separieren, kann das Coordinator-Pattern genutzt werden. Hierbei instanziiert die View ein eigenes Objekt, welches sie durch die \texttt{@StateObject} Annotation besitzt und kontrollieren kann, bspw. durch die Interaktion des Nutzers oder beim Erscheinen der Ansicht. Das Coordinator-Objekt macht die darzustellenden Datenattribute über die \texttt{@Published} Annotation nach außen verfügbar - so dass die View diese Datenattribute (ähnlich zu \texttt{@State} Datenattributen) verwenden und auf deren Änderungen reagieren kann.

\begin{figure}[H]
\includegraphics[width=\linewidth, bb=0 0 360 253]{coordinator.pdf}
\caption{Das Coordinator Pattern illustriert am Beispiel eines hypothetischen QR Code Scanners: Der Coordinator realisiert die Initialisierung des Scanners und die Validierung des Codes. Die View stellt den Code lediglich dar, sobald er gescannt wurde.}\label{fig:coordinator}
\end{figure}

In \Cref{fig:coordinator} ist das Pattern am Beispiel illustriert. Auch zu sehen ist hierbei, dass der Coordinator eine Klasse ist. Dies ist nützlich wenn bestimmte Protokolle (Swift-Interfaces) implementiert werden müssen, in diesem Beispiel \texttt{AVCaptureMetadataOutputObjectsDelegate} zur Reaktion auf gescannte QR Codes. In diesem Beispiel kann die View dieses Protokoll nicht implementieren, da das Protokoll implizit die Implementation durch eine Klasse erfordert.

\subsection{Environment-Pattern}

\begin{figure}[H]
\includegraphics[width=\linewidth, bb=0 0 275 167]{environment.pdf}
\caption{Das Environment Pattern illustriert am Beispiel der Map View: Die Map View erstellt ihr eigenes Environment und gibt dieses an ihre Unteransichten weiter. Somit kann die gewählte Etage von einer Unteransicht modifiziert werden, von anderen Unteransichten wird sie lediglich dargestellt.}\label{fig:coordinator}
\end{figure}

\subsection{MVVM-Pattern}

\subsection{Proxy-Pattern}

\subsectino{Eventbasierte Datenflüsse}
