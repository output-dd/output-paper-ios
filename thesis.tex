% the following command is only required if the thesis is written in german
\RequirePackage[ngerman=ngerman-x-latest]{hyphsubst}

\documentclass[
  ngerman, % change to ngerman for german theses
  symmetric, % use two-side for booklike layouts
  numbers=noenddot % remove trailing dots in chapter/section/... enumeration
]{tudscrreprt}

\usepackage[T1]{fontenc}
\usepackage[utf8]{inputenc}
\usepackage[
  ngerman % change to ngerman for german theses
]{babel}
\usepackage{isodate}
\usepackage{pdfpages}
\usepackage{listings}
\usepackage[toc, page]{appendix}
\usepackage{hyphenat}

\usepackage[
  style=alphabetic,
  backend=biber,
  url=false,
  doi=false,
  isbn=false,
  hyperref,
]{biblatex}
% configure the location of the biblatex file
\addbibresource{bibliography.bib}
\AtEveryBibitem{%
  \clearfield{note}%
}

% make all links clickable but hide ugly boxes
\usepackage[hidelinks]{hyperref}
% automatically insert Fig. X in the text with \cref{..}
\usepackage[capitalise,nameinlink,noabbrev]{cleveref}

\usepackage[colorinlistoftodos,prependcaption,textsize=tiny]{todonotes}

\usepackage{graphicx}
\graphicspath{ {./images/} }

\usepackage{svg}

% if you need mathy stuff
\newtheorem{lem}{Lemma}
\crefname{lem}{Lemma}{Lemmas}
\newtheorem{thm}{Theorem}
\crefname{thm}{Theorem}{Theorems}
\newtheorem{defs}{Definition}
\crefname{defs}{Def.}{Defs.}

\usepackage{blindtext}

%\usepackage{tudscrsupervisor} % if you want to copy the sources of the task description into the thesis

\usepackage{csquotes}

\usepackage{caption}
\captionsetup{font=normalfont,labelfont=normalfont,labelsep=space}
\usepackage{floatrow}
\floatsetup{font=normalfont}
\floatsetup[table]{style=plaintop}
\captionsetup{singlelinecheck=off,format=hang,justification=raggedright}
\DeclareCaptionSubType[alph]{figure}
\DeclareCaptionSubType[alph]{table}
\captionsetup[subfloat]{labelformat=brace,list=off}

\usepackage{booktabs}
\usepackage{array}
\usepackage{tabularx}
\usepackage{tabulary}
\usepackage{tabu}
\usepackage{longtable}
\usepackage{multirow}

\usepackage{quoting}

\usepackage[babel]{microtype}

\usepackage{xfrac}

\usepackage{enumitem}
\setlist[itemize]{noitemsep}

\usepackage{ellipsis}
\let\ellipsispunctuation\relax

\usepackage{listings}
\usepackage{xcolor}

\definecolor{commentsColor}{rgb}{0.497495, 0.497587, 0.497464}
\definecolor{keywordsColor}{rgb}{0.000000, 0.000000, 0.635294}
\definecolor{stringColor}{rgb}{0.558215, 0.000000, 0.135316}

\lstset{ %
  backgroundcolor=\color{white},   % choose the background color; you must add \usepackage{color} or \usepackage{xcolor}
  basicstyle=\footnotesize,        % the size of the fonts that are used for the code
  breakatwhitespace=false,         % sets if automatic breaks should only happen at whitespace
  breaklines=true,                 % sets automatic line breaking
  captionpos=b,                    % sets the caption-position to bottom
  commentstyle=\color{commentsColor}\textit,    % comment style
  deletekeywords={...},            % if you want to delete keywords from the given language
  escapeinside={\%*}{*)},          % if you want to add LaTeX within your code
  extendedchars=true,              % lets you use non-ASCII characters; for 8-bits encodings only, does not work with UTF-8
  frame=tb,	                   	   % adds a frame around the code
  keepspaces=true,                 % keeps spaces in text, useful for keeping indentation of code (possibly needs columns=flexible)
  keywordstyle=\color{keywordsColor}\bfseries,       % keyword style
  language=Python,                 % the language of the code (can be overrided per snippet)
  otherkeywords={*,...},           % if you want to add more keywords to the set
  numbers=left,                    % where to put the line-numbers; possible values are (none, left, right)
  numbersep=5pt,                   % how far the line-numbers are from the code
  numberstyle=\tiny\color{commentsColor}, % the style that is used for the line-numbers
  rulecolor=\color{black},         % if not set, the frame-color may be changed on line-breaks within not-black text (e.g. comments (green here))
  showspaces=false,                % show spaces everywhere adding particular underscores; it overrides 'showstringspaces'
  showstringspaces=false,          % underline spaces within strings only
  showtabs=false,                  % show tabs within strings adding particular underscores
  stepnumber=1,                    % the step between two line-numbers. If it's 1, each line will be numbered
  stringstyle=\color{stringColor}, % string literal style
  tabsize=2,	                   % sets default tabsize to 2 spaces
  title=\lstname,                  % show the filename of files included with \lstinputlisting; also try caption instead of title
  columns=fixed                    % Using fixed column width (for e.g. nice alignment)
}

\lstdefinelanguage{XML} % use with language = XML
{
  morestring=[b]",
  morestring=[s]{>}{<},
  morecomment=[s]{<?}{?>},
  morekeywords={xmlns,version,type}
}


 % code styles (listings)

% use this custom theorem for research questions
\newtheorem{researchquestion}{Forschungsfrage}
\crefname{researchquestion}{Forschungsfrage}{Forschungsfragen}

% use this custom environment for equations
\newenvironment{conditions}
  {\par\vspace{\abovedisplayskip}\noindent\begin{tabular}{>{$}l<{$} @{${}={}$} l}}
  {\end{tabular}\par\vspace{\belowdisplayskip}}

\usepackage{float}

% configure the name of your appendix
\renewcommand\appendixtocname{Anhang}
\renewcommand\appendixpagename{Anhang}

% use \tocless before a chapter/section/... in the
% appendix to hide it from the toc
\newcommand{\nocontentsline}[3]{}
\newcommand{\tocless}[2]{
  \bgroup\let\addcontentsline=\nocontentsline#1{#2}\egroup
}

\begin{document}

  % use uppercase roman letters for all pages until the introduction
  % this way, it is easier to identify how many pages the thesis has
  \pagenumbering{Roman}

  \faculty{Fakultät Informatik}
  \department{}
  \institute{Institut für Systemarchitektur}
  \chair{Professur für Rechnernetze}
  \title{%
    Dokumentation über den Neuaufbau der OUTPUT.DD App im WiSe 2020/21
  }

  \thesis{project} % the type of thesis you want to write

  \author{Philipp Matthes}
  \matriculationnumber{4605459}
  \matriculationyear{WiSe 2016/17}
  \dateofbirth{12.3.1997}
  \placeofbirth{Chemnitz}

  \course{Diplom Informatik (PO 2010)}

  \supervisor{%
    Dr. Thomas Springer
  }
  \date{2. Februar 2020} % the date of submission
  \maketitle

  \newpage

  % include the task definition if you want
  % \includepdf[pages=-]{task/task.pdf}
  % \newpage

  % for the order of the following sections please refer to
  % the recommendations for thesis structuring
  \confirmation

  \tableofcontents

  \listoffigures
  \addcontentsline{toc}{chapter}{\listfigurename}

  \listoftables
  \addcontentsline{toc}{chapter}{\listtablename}

  % this is where your thesis lives
  \chapter{Einleitung}\label{ch:einleitung}\pagenumbering{arabic}

OUTPUT.DD ist eine jährlich stattfindende Projektschau, auf welcher in der Fakultät Informatik der Technischen Universität Dresden wissenschaftliche Projekte von Studenten, Firmen und Mitarbeitern präsentiert und ausgestellt werden. Begleitend zur Projektschau wird den Besuchern der Veranstaltung jeweils eine App für iOS und Android zur Verfügung gestellt. Die Apps dienen dabei selbst nicht nur als Plattform, über welche Nutzer den zeitlichen und topografischen Veranstaltungsplan einsehen können, sondern auch als Integrationspunkt für verschiedene Forschungsprojekte aus den Bereichen Mediengestaltung, Application Development und Mobile Computing. So wird die Interaktion der Nutzer beispielsweise durch eine Gamification\footnote{Gamification. Einsatz von Spielelementen in einem Nicht-Spiel-Kontext.} motiviert und geleitet, oder die Bewegung des Nutzers auf der Messe über ein im Rahmen des Forschungsprojektes \enquote{Mapbiquitous} entstandenes Beaconing-Framework verfolgt, um das Besucheraufkommen zu analysieren und dem Nutzer eine interaktive Heatmap auf der Kartenansicht der Veranstaltung anzuzeigen. Außerdem nutzt die App das Offline-First-Prinzip aus dem Forschungsbereich Ubiquitäre Applikationen, bei dem die in der App angezeigten Daten koordiniert von einem Server-Backend synchronisiert werden, um bei einem Netzwerkausfall weiterhin so viele Funktionalitäten wie möglich zu unterstützen und die offline geschriebenen Daten bei der Wiederherstellung der Verbindung im Hintergrund zu aktualisieren. Im Sommer 2020 wurden für die beiden Apps jeweils Datenbank-Frameworks für die NoSQL-Datenbank Couchbase entwickelt, welche architekturell im Repository-Pattern\footnote{Repository-Pattern. Architekturelles Pattern zur Abstraktion und Separation von Datenbankabfragen durch die Bereitstellung von CRUD-Operationen (Create, Read, Update, Delete).} umgesetzt wurden und zur Synchronisation der Daten so genannte Replikatoren bereitstellen, die über einen Synchronisationsdienst mit dem Couchbase Backend-Server kommunizieren.

\section{Problemstellung}

Da die OUTPUT.DD Apps in den vergangenen Jahren noch auf ein anderes, auf Realm basierendes, Datenbank-Backend aufsetzten, musste eine Substitution des Datenbank-Backends analysiert, geplant, durchgeführt und getestet werden.

Durch die sukzessive funktionelle Erweiterung der Apps über mehrere Jahre unter verschiedenen Teams mit jeweils unterschiedlichen Qualitätsansprüchen, Zeitvorgaben und Kenntnisständen bildeten sich außerdem mehrere Probleme in der Implementation heraus, die durch einen Audit der iOS-Codebasis am 5.4.2020 identifiziert werden konnten:

\begin{itemize}
  \item \textbf{Strukturelle Antipatterns}, darunter eine schwer nachvollziehbare Ordner- und Dateistrukturierung, das Nichtvorhandensein einer Separation in Module, sowie die Vermischung von abkapselbaren externen und internen Frameworks mit der primären Codebasis
  \item \textbf{Architekturelle Antipatterns}, wie beispielsweise die \enquote{Verschmutzung}\footnote{Verschmutzung. Von Englisch \enquote{Pollution}, wird oft als Fachbegriff verwendet, um die Degradation der Codequalität durch fehlpositionierte Codefragmente zu verbildlichen.} des Codes durch globale Erweiterungen von Datenbankmodellen an unerwarteten Stellen
  \item \textbf{Weitere Probleme}, darunter die Ignorierung von Sicherheitsrisiken durch das Ausschalten von Dependency-Warnings, an einer Stelle auch die Fehlverwendung von View-Life-Cycles, das allgemeine Vorhandensein Code Smells und die technische Alterung des Objective-C Codes.
\end{itemize}

\section{Ziele der Projektarbeit}

Da OUTPUT.DD 2020 vor dem Hintergrund der Sars-CoV-2 Pandemie nicht stattfand, wurde im Projektteam beschlossen, diese Probleme durch eine Neuimplementation der Applikation zu beheben, mit dem beiläufigen Ziel, auch das neue Datenbank-Backend zu integrieren. Im Projektteam wurde entschieden, die Apps mit den jeweils aktuellsten Technologien neu zu implementieren, darunter eine Reimplementation der Android-App mit der Programmiersprache Kotlin, sowie eine (in dieser Arbeit näher beschriebenen) Reimplementation der iOS-App mit der Programmiersprache Swift und dem 2019 von Apple eingeführten UI-Framework SwiftUI, welches eine deklarative Ansichtserstellung ermöglicht. Hierdurch sollten die Apps technisch und visuell modernisiert werden. Bei der Reimplementation sollten die oben beschriebenen Probleme durch die Einführung klarer Architektur- und Strukturpatterns möglichst vermieden werden, sowie außerdem konkrete Strategien und Richtlinien entworfen werden, um ein Wiedereintreten dieser Probleme zu vermeiden und somit zukünftigen Projektteams die Weiterentwicklung der Apps zu erleichtern. Diese Arbeit soll einen chronologischen Überblick darüber geben, welche Schritte gegangen wurden, um diese Ziele zu erreichen, und die Endprodukte schließlich evaluieren.


  \newpage

  % use lowercased roman page numbers for the appendix and the bibliography
  \pagenumbering{roman}

  \printbibliography[heading=bibintoc]\label{sec:bibliography}%

  \begin{appendices}
    \tocless\chapter{Appendix A}\label{appendix:a}

  \end{appendices}

\end{document}
